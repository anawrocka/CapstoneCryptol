\section{Layout}\label{layout}

Groups of declarations are organized based on indentation. Declarations
with the same indentation belong to the same group. Lines of text that
are indented more than the beginning of a declaration belong to that
declaration, while lines of text that are indented less terminate a
group of declaration. Groups of declarations appear at the top level of
a Cryptol file, and inside \texttt{where} blocks in expressions. For
example, consider the following declaration group

\begin{verbatim}
f x = x + y + z
  where
  y = x * x
  z = x + y

g y = y
\end{verbatim}

This group has two declaration, one for \texttt{f} and one for
\texttt{g}. All the lines between \texttt{f} and \texttt{g} that are
indented more then \texttt{f} belong to \texttt{f}.

This example also illustrates how groups of declarations may be nested
within each other. For example, the \texttt{where} expression in the
definition of \texttt{f} starts another group of declarations,
containing \texttt{y} and \texttt{z}. This group ends just before
\texttt{g}, because \texttt{g} is indented less than \texttt{y} and
\texttt{z}.

\section{Comments}\label{comments}

Cryptol supports block comments, which start with \texttt{/*} and end
with \texttt{*/}, and line comments, which start with \texttt{//} and
terminate at the end of the line. Block comments may be nested
arbitrarily.

Examples:

\begin{verbatim}
/* This is a block comment */
// This is a line comment
/* This is a /* Nested */ block comment */
\end{verbatim}

\section{Identifiers}\label{identifiers}

Cryptol identifiers consist of one or more characters. The first
character must be either an English letter or underscore (\texttt{\_}).
The following characters may be an English letter, a decimal digit,
underscore (\texttt{\_}), or a prime (\texttt{'}). Some identifiers have
special meaning in the language, so they may not be used in
programmer-defined names (see
\hyperref[keywords-and-built-in-operators]{Keywords}).

Examples:

\begin{verbatim}
name    name1    name'    longer_name
Name    Name2    Name''   longerName
\end{verbatim}

\hyperdef{}{keywords-and-built-in-operators}{\section{Keywords and
Built-in Operators}\label{keywords-and-built-in-operators}}

The following identifiers have special meanings in Cryptol, and may not
be used for programmer defined names:

\begin{verbatim}
Arith   Inf       extern    inf    module      then     
Bit     True      fin       lg2    newtype     type     
Cmp     else      if        max    pragma      where    
False   export    import    min    property    width    
\end{verbatim}

The following table contains Cryptol's operators and their associativity
with lowest precedence operators first, and highest precedence last.

\begin{longtable}[c]{@{}ll@{}}
\toprule\addlinespace
Operator & Associativity
\\\addlinespace
\midrule\endhead
\texttt{\textbar{}\textbar{}} & left
\\\addlinespace
\texttt{\^{}} & left
\\\addlinespace
\texttt{\&\&} & left
\\\addlinespace
\texttt{-\textgreater{}} (types) & right
\\\addlinespace
\texttt{!=} \texttt{==} & not associative
\\\addlinespace
\texttt{\textgreater{}} \texttt{\textless{}} \texttt{\textless{}=}
\texttt{\textgreater{}=} & not associative
\\\addlinespace
\texttt{\#} & right
\\\addlinespace
\texttt{\textgreater{}\textgreater{}} \texttt{\textless{}\textless{}}
\texttt{\textgreater{}\textgreater{}\textgreater{}}
\texttt{\textless{}\textless{}\textless{}} & left
\\\addlinespace
\texttt{+} \texttt{-} & left
\\\addlinespace
\texttt{*} \texttt{/} \texttt{\%} & left
\\\addlinespace
\texttt{\^{}\^{}} & right
\\\addlinespace
\texttt{!} \texttt{!!} \texttt{@} \texttt{@@} & left
\\\addlinespace
(unary) \texttt{-} \texttt{\textasciitilde{}} & right
\\\addlinespace
\bottomrule
\addlinespace
\caption{Operator precedences.}
\end{longtable}

\section{Numeric Literals}\label{numeric-literals}

Numeric literals may be written in binary, octal, decimal, or
hexadecimal notation. The base of a literal is determined by its prefix:
\texttt{0b} for binary, \texttt{0o} for octal, no special prefix for
decimal, and \texttt{0x} for hexadecimal.

Examples:

\begin{verbatim}
254                 // Decimal literal
0254                // Decimal literal
0b11111110          // Binary literal
0o376               // Octal literal
0xFE                // Hexadecimal literal
0xfe                // Hexadecimal literal
\end{verbatim}

Numeric literals represent finite bit sequences (i.e., they have type
\texttt{{[}n{]}}). Using binary, octal, and hexadecimal notation results
in bit sequences of a fixed length, depending on the number of digits in
the literal. Decimal literals are overloaded, and so the length of the
sequence is inferred from context in which the literal is used.
Examples:

\begin{verbatim}
0b1010              // : [4],   1 * number of digits
0o1234              // : [12],  3 * number of digits
0x1234              // : [16],  4 * number of digits

10                  // : {n}. (fin n, n >= 4) => [n]
                    //   (need at least 4 bits)

0                   // : {n}. (fin n) => [n]
\end{verbatim}

\section{Bits}\label{bits}

The type \texttt{Bit} has two inhabitants: \texttt{True} and
\texttt{False}. These values may be combined using various logical
operators, or constructed as results of comparisons.

\begin{longtable}[c]{@{}lll@{}}
\toprule\addlinespace
Operator & Associativity & Description
\\\addlinespace
\midrule\endhead
\texttt{\textbar{}\textbar{}} & left & Logical or
\\\addlinespace
\texttt{\^{}} & left & Exclusive-or
\\\addlinespace
\texttt{\&\&} & left & Logical and
\\\addlinespace
\texttt{!=} \texttt{==} & none & Not equals, equals
\\\addlinespace
\texttt{\textgreater{}} \texttt{\textless{}} \texttt{\textless{}=}
\texttt{\textgreater{}=} & none & Comparisons
\\\addlinespace
\texttt{\textasciitilde{}} & right & Logical negation
\\\addlinespace
\bottomrule
\addlinespace
\caption{Bit operations.}
\end{longtable}

\section{If Then Else with Multiway}\label{if-then-else-with-multiway}

\texttt{If then else} has been extended to support multi-way
conditionals. Examples:

\begin{verbatim}
x = if y % 2 == 0 then 22 else 33

x = if y % 2 == 0 then 1
     | y % 3 == 0 then 2
     | y % 5 == 0 then 3
     else 7
\end{verbatim}

\section{Tuples and Records}\label{tuples-and-records}

Tuples and records are used for packaging multiples values together.
Tuples are enclosed in parenthesis, while records are enclosed in
braces. The components of both tuples and records are separated by
commas. The components of tuples are expressions, while the components
of records are a label and a value separated by an equal sign. Examples:

\begin{verbatim}
(1,2,3)           // A tuple with 3 component
()                // A tuple with no components

{ x = 1, y = 2 }  // A record with two fields, `x` and `y`
{}                // A record with no fileds
\end{verbatim}

The components of tuples are identified by position, while the
components of records are identified by their label, and so the ordering
of record components is not important. Examples:

\begin{verbatim}
           (1,2) == (1,2)               // True
           (1,2) == (2,1)               // False

{ x = 1, y = 2 } == { x = 1, y = 2 }    // True
{ x = 1, y = 2 } == { y = 2, x = 1 }    // True
\end{verbatim}

The components of a record or a tuple may be accessed in two ways: via
pattern matching or by using explicit component selectors. Explicit
component selectors are written as follows:

\begin{verbatim}
(15, 20).0           == 15
(15, 20).1           == 20

{ x = 15, y = 20 }.x == 15
\end{verbatim}

Explicit record selectors may be used only if the program contains
sufficient type information to determine the shape of the tuple or
record. For example:

\begin{verbatim}
type T = { sign :: Bit, number :: [15] }

// Valid defintion:
// the type of the record is known.
isPositive : T -> Bit
isPositive x = x.sign

// Invalid defintion:
// insufficient type information.
badDef x = x.f
\end{verbatim}

The components of a tuple or a record may also be access by using
pattern matching. Patterns for tuples and records mirror the syntax for
constructing values: tuple patterns use parenthesis, while record
patterns use braces. Examples:

\begin{verbatim}
getFst (x,_) = x

distance2 { x = xPos, y = yPos } = xPos ^^ 2 + yPos ^^ 2

f x = fst + snd where
\end{verbatim}

\section{Sequences}\label{sequences}

A sequence is a fixed-length collection of element of the same type. The
type of a finite sequence of length \texttt{n}, with elements of type
\texttt{a} is \texttt{{[}n{]} a}. Often, a finite sequence of bits,
\texttt{{[}n{]} Bit}, is called a \emph{word}. We may abbreviate the
type \texttt{{[}n{]} Bit} as \texttt{{[}n{]}}. An infinite sequence with
elements of type \texttt{a} has type \texttt{{[}inf{]} a}, and
\texttt{{[}inf{]}} is an infinite stream of bits.

\begin{verbatim}
[e1,e2,e3]                        // A sequence with three elements

[t .. ]                           // Sequence enumerations
[t1, t2 .. ]                      // Step by t2 - t1
[t1 .. t3 ]
[t1, t2 .. t3 ]
[e1 ... ]                         // Infinite sequence starting at e1
[e1, e2 ... ]                     // Infinite sequence stepping by e2-e1

[ e | p11 <- e11, p12 <- e12      // Sequence comprehensions
    | p21 <- e21, p22 <- e22 ]
\end{verbatim}

Note: the bounds in finite unbounded (those with ..) sequences are type
expressions, while the bounds in bounded-finite and infinite sequences
are value expressions.

\begin{longtable}[c]{@{}ll@{}}
\toprule\addlinespace
Operator & Description
\\\addlinespace
\midrule\endhead
\texttt{\#} & Sequence concatenation
\\\addlinespace
\texttt{\textgreater{}\textgreater{}} \texttt{\textless{}\textless{}} &
Shift (right,left)
\\\addlinespace
\texttt{\textgreater{}\textgreater{}\textgreater{}}
\texttt{\textless{}\textless{}\textless{}} & Rotate (right,left)
\\\addlinespace
\texttt{@} \texttt{!} & Access elements (front,back)
\\\addlinespace
\texttt{@@} \texttt{!!} & Access sub-sequence (front,back)
\\\addlinespace
\bottomrule
\addlinespace
\caption{Sequence operations.}
\end{longtable}

There are also lifted point-wise operations.

\begin{verbatim}
[p1, p2, p3, p4]          // Sequence pattern
p1 # p2                   // Split sequence pattern
\end{verbatim}

\section{Functions}\label{functions}

\begin{verbatim}
\p1 p2 -> e              // Lambda expression
f p1 p2 = e              // Function definition
\end{verbatim}

\section{Local Declarations}\label{local-declarations}

\begin{verbatim}
e where ds
\end{verbatim}

Note that by default, any local declarations without type signatures
are monomorphized. If you need a local declaration to be polymorphic,
use an explicit type signature.

\section{Explicit Type Instantiation}\label{explicit-type-instantiation}

If \texttt{f} is a polymorphic value with type:

\begin{verbatim}
f : { tyParam }

f `{ tyParam = t }
\end{verbatim}

\section{Demoting Numeric Types to
Values}\label{demoting-numeric-types-to-values}

The value corresponding to a numeric type may be accessed using the
following notation:

\begin{verbatim}
`{t}
\end{verbatim}

Here \texttt{t} should be a type expression with numeric kind. The
resulting expression is a finite word, which is sufficiently large to
accomodate the value of the type:

\begin{verbatim}
`{t} :: {w >= width t}. [w]
\end{verbatim}

\section{Explicit Type Annotations}\label{explicit-type-annotations}

Explicit type annotations may be added on expressions, patterns, and in
argument definitions.

\begin{verbatim}
e : t

p : t

f (x : t) = ...
\end{verbatim}

\section{Type Signatures}\label{type-signatures}

\begin{verbatim}
f,g : {a,b} (fin a) => [a] b
\end{verbatim}

\section{Type Synonym Declarations}\label{type-synonym-declarations}

\begin{verbatim}
type T a b = [a] b
\end{verbatim}
